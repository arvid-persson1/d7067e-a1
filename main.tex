\documentclass{article}
\usepackage[utf8]{inputenc}
\usepackage{amsmath, amsthm, amssymb}
\usepackage[a4paper, includeheadfoot, margin=2.54cm]{geometry}

\renewcommand\thesection{\arabic{section}.}

\title{Assignment 1}
\author{Shaya Rezai Yazdi \and Arvid Persson}

\begin{document}

\maketitle

\section{}

\begin{proof}
    The proof is by induction. Let the list be $S$ with $|S| = n < \infty$.

    For $n = 0$, the empty tree is a treap. For $n = 1$, the single node is a
    treap. Assume there exists exactly one treap for any list smaller than $S$
    with the same properties.

    As the priorities are distinct, there exists a $r = (k_0, p_0) \in S$ such
    that $\forall (k, p) \in S \setminus r: p_0 > p$. Let $L = \{ (k, p) \in S:
    k < k_0 \}, R = \{ (k, p) \in S: k > k_0 \}$, meaning $S = L \cup \{r\}
    \cup R$. Consider the treaps constructed from $L, R$ respectively: By the
    inductive hypothesis, since $L, R \subset S$, these exist and are unique.

    Construct the tree with $r$ as the root, satisfying the heap property, and
    $L, R$ as its left and right subtrees respectively, satisfying the binary
    tree property. As any subtree of a treap is itself a treap, this tree is
    unique, and by $p_0$ being the highest priority, a treap.
\end{proof}

\section{}

Let $A(i, j)$ determine whether $k_i$ is an ancestor of $k_j$ in $T$. An
inorder traversal of a binary tree produces a sorted list. As such, for $k_i$
to be an ancestor of $k_j$, it must be the highest priority key in the
inclusive interval between them in the list. There are $|i - j| + 1$ such
nodes, each equally likely to have the highest priority. If we define a node to
never be an ancestor of itself, we find

\begin{equation*}
    \Pr(A(i, j)) =
    \begin{cases}
	0 & i = j, \\
	\frac{1}{|i - j| + 1} & i \neq j.
    \end{cases}
\end{equation*}

\end{document}
